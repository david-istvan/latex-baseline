\documentclass{article}
\usepackage[utf8]{inputenc}
\usepackage{ifthen}
\usepackage{hyperref}

\usepackage[dvipsnames]{xcolor}
%\PassOptionsToPackage{dvipsnames}{xcolor} % <-- use this if \usepackage[dvipsnames]{xcolor} clashes with the chose template (e.g., acmart)
\usepackage{soul}
\usepackage[normalem]{ulem}
\usepackage{graphicx}

\usepackage{epsf,picinpar}
\usepackage{varioref}
\usepackage{fdsymbol}

\usepackage{enumitem}

\usepackage[numbers,sort&compress,comma,square]{natbib}

\usepackage{pifont}% http://ctan.org/pkg/pifont
\usepackage[pscoord]{eso-pic}
%%%%%%%%%%%%%%%%%%% Basic commands for editing %%%%%%%%%%%%%%%%%%%
\newcommand{\secref}[1]{Section~\ref{#1}}
\newcommand{\chpref}[1]{Chapter~\ref{#1}}
\newcommand{\figref}[1]{Figure~\ref{#1}}
\newcommand{\tabref}[1]{Table~\ref{#1}}
\newcommand{\equref}[1]{Equation~\ref{#1}}
\newcommand{\lstref}[1]{Listing~\ref{#1}}
\newcommand{\appref}[1]{Appendix~\ref{#1}}

%%%%%%%%%%%%%%%%%%% Frequently used annotations %%%%%%%%%%%%%%%%%%%

\newcommand{\todo}[1]{
    \ifthenelse{\boolean{showannotations}}%
    {\ifthenelse{\equal{#1}{}}{\textcolor{red}{TODO}}{\textcolor{red}{TODO:~{#1}}}}%
    {}%
}

\newcommand{\assignedto}[1]{%
    \ifthenelse{\boolean{showannotations}}%
    {\textbf{\noindent\ding{46}\textcolor{white}{~\colorbox{\assignementcolor}{Assigned to:}}~\textcolor{\assignementcolor}{#1}\\}%
    }
    {}
}

%%%%%%%%%%%%%%%%%%% Review commands %%%%%%%%%%%%%%%%%%%

\newcommand{\rephrase}[1]{\textcolor{purple}{\uwave{#1}}} % please rephrase

\renewcommand{\nb}[4]{
    \ifthenelse{\boolean{showannotations}}%
    {\fcolorbox{gray}{#2}{\bfseries\sffamily\scriptsize{#1}}
	{\sf\small$\blacktriangleright$\textcolor{#4}{\textit{#3}}$\blacktriangleleft$}}%
    {}%
}

\newcommand\personmarker[2]{\noindent\nb{#1}{yellow}{#2}{VioletRed}}
\newcommand\id[1]{\noindent\personmarker{ID}{#1}}

\newcommand\re[2]{\noindent\nb{#1}{YellowOrange}{#2}{RubineRed}}

%%%%%%%%%%%%%%%%%%% Revision commands %%%%%%%%%%%%%%%%%%%
\newcommand{\rem}[1]{%
    \ifthenelse{\boolean{showannotations}}%
    {\textcolor{\oldtextcolor}{\st{#1}}}%
    {}%
}

\newcommand\add[1]{%
    \ifthenelse{\boolean{showannotations}}%
    {\textcolor{\newtextcolor}{{#1}}}%
    {#1}%
}

\newcommand\rep[2]{%
    \ifthenelse{\boolean{showannotations}}%
    {\rem{#1}~\add{#2}}%
    {#2}%
}
%%%%%%%%%%%%%%%%%%% Show/hide annotations %%%%%%%%%%%%%%%%%%%
\newboolean{showannotations}
\setboolean{showannotations}{true} % TOGGLE TO SHOW ANNOTATIONS

%%%%%%%%%%%%%%%%%%% Show/hide names for double blind reviews %%%%%%%%%%%%%%%%%%%
\newboolean{shownames}
\setboolean{shownames}{true}


%%%%%%%%%%%%%%%%%%% Colors %%%%%%%%%%%%%%%%%%%
\newcommand{\newtextcolor}{blue}
\newcommand{\oldtextcolor}{red}
\newcommand{\assignementcolor}{orange}
\definecolor{highlightcolor}{rgb}{.99, 1, .0}
\sethlcolor{highlightcolor}

%%%%%%%%%%%%%%%%%%% Custom formatting (if allowed) %%%%%%%%%%%%%%%%%%%
\setlength\parindent{0pt}
\usepackage[margin=1in, top = 1in, bottom = 0.8in, headheight = 0.3in]{geometry}

\title{latex-baseline}
\author{
    \ifthenelse{\boolean{shownames}}%
    {Istvan David \id{The name is shown if the "shownames" option is set to true}}%
    {Anonymous author(s) \id{Authors are anonymized if the "shownames" option is set to false}}%
}
\date{April 2024}

\begin{document}

\preprintlabel{xx.xxxx/xxxx}{Here's an example of pre-print labeling. Use the \textbackslash\texttt{preprintlabel} command.}

\maketitle

\section{Structure}
\begin{itemize}
    \item \verb|Preamble|
    \begin{itemize}
        \item \verb|packages.tex|: packages are imported here.
        \item \verb|commands.tex|: commands are defined here.
        \item \verb|settings.tex|: settings are defined here.
    \end{itemize}
\end{itemize}

\section{Settings}

The settings can be updated in the \verb|settings.tex| file.

\subsection{Margins}
Update \verb|\usepackage[margin=1in, top = 1in, bottom = 0.8in, headheight = 0.3in]{geometry}| as needed.

Update \verb|\setlength\parindent{0pt}| as needed.


\subsection{Anonymize paper for double blind reviews}
Replace \verb|\author{Steve}| with the following:
\begin{verbatim}
\author{
    \ifthenelse{\boolean{shownames}}%
    {Steve}%
    {Anonymous author(s)}%
}
\end{verbatim}

Use this block for other sensitive information. Change the \verb|\setboolean{shownames}{true}| setting to \verb|\setboolean{shownames}{false}|.

\subsection{Colors}

The commands in \secref{sec:commands} often use colors. These can be changed by modifying the following settings.
\begin{itemize}
    \item \verb|\newcommand{\newtextcolor}{blue}| -- sets the color of the added text annotations.
    \item \verb|\newcommand{\assignementcolor}{blue}| -- sets the color of the assignment annotations.
\end{itemize}





\section{Commands}\label{sec:commands}

\subsection{Frequently used annotations}

\begin{itemize}
    \item This is normal text.
    \item \todo{This is a todo.}
    \item \todo{} (Empty todo label.)
    \item \assignedto{Steve}
\end{itemize}

\subsection{Review commands}

\begin{itemize}
    \item \rephrase{Should be rephrased}
    \item \id{This is my comment with my initials}
    \item \re{R1.01}{This marks a comment related to a reviewer's specific remark}
    \item \hl{This is a highlighting from the \texttt{soul} package for anything else.}
\end{itemize}

\subsection{Revision annotations}

\begin{itemize}
    \item \rem{Text removed.}
    \item \add{Text added.}
    \item \rep{Text replaced.}{New text.}
    \item \rem{Text with a reference{~\cite{bostrom2003are}} removed. -- Use braces around \texttt{$\sim$\textbackslash cite}.}
    \item \add{The \texttt{\textbackslash add\{\}} command does \textbf{NOT} support multiple paragraphs.}
    
\end{itemize}

Some journals require submitting camera-ready revisions, but attaching the annotated version is advisable to enable traceability between the rebuttal and the revised manuscript, thus expiding things. Change the \verb|\setboolean{showannotations}{true}| setting to \verb|\setboolean{showannotations}{false}| to hide \verb|\add{}|, \verb|\rem{}| and \verb|\rep{}| annotations and produce the final version.

\bibliographystyle{ieeetr}
\bibliography{bib/references}

\end{document}
