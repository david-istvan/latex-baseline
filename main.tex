\documentclass{article}
\usepackage[utf8]{inputenc}
\usepackage{ifthen}
\usepackage{xcolor}
\usepackage{soul}
\usepackage[normalem]{ulem}

\usepackage{epsf,picinpar}
\usepackage{varioref}
\usepackage{fdsymbol}

\usepackage{pifont}% http://ctan.org/pkg/pifont
%%%%%%%%%%%%%%%%%%% Basic commands for editing %%%%%%%%%%%%%%%%%%%
\newcommand{\secref}[1]{Section~\ref{#1}}
\newcommand{\chpref}[1]{Chapter~\ref{#1}}
\newcommand{\figref}[1]{Figure~\ref{#1}}
\newcommand{\tabref}[1]{Table~\ref{#1}}
\newcommand{\equref}[1]{Equation~\ref{#1}}
\newcommand{\lstref}[1]{Listing~\ref{#1}}
\newcommand{\appref}[1]{Appendix~\ref{#1}}

%%%%%%%%%%%%%%%%%%% Frequently used annotations %%%%%%%%%%%%%%%%%%%

\newcommand{\todo}[1]{
    \ifthenelse{\boolean{showannotations}}%
    {\ifthenelse{\equal{#1}{}}{\textcolor{red}{TODO}}{\textcolor{red}{TODO:~{#1}}}}%
    {}%
}

\newcommand{\assignedto}[1]{%
    \ifthenelse{\boolean{showannotations}}%
    {\textbf{\noindent\ding{46}\textcolor{white}{~\colorbox{\assignementcolor}{Assigned to:}}~\textcolor{\assignementcolor}{#1}\\}%
    }
    {}
}

%%%%%%%%%%%%%%%%%%% Review commands %%%%%%%%%%%%%%%%%%%

\newcommand{\rephrase}[1]{\textcolor{purple}{\uwave{#1}}} % please rephrase

\renewcommand{\nb}[4]{
    \ifthenelse{\boolean{showannotations}}%
    {\fcolorbox{gray}{#2}{\bfseries\sffamily\scriptsize{#1}}
	{\sf\small$\blacktriangleright$\textcolor{#4}{\textit{#3}}$\blacktriangleleft$}}%
    {}%
}

\newcommand\personmarker[2]{\noindent\nb{#1}{yellow}{#2}{VioletRed}}
\newcommand\id[1]{\noindent\personmarker{ID}{#1}}

\newcommand\re[2]{\noindent\nb{#1}{YellowOrange}{#2}{RubineRed}}

%%%%%%%%%%%%%%%%%%% Revision commands %%%%%%%%%%%%%%%%%%%
\newcommand{\rem}[1]{%
    \ifthenelse{\boolean{showannotations}}%
    {\textcolor{\oldtextcolor}{\st{#1}}}%
    {}%
}

\newcommand\add[1]{%
    \ifthenelse{\boolean{showannotations}}%
    {\textcolor{\newtextcolor}{{#1}}}%
    {#1}%
}

\newcommand\rep[2]{%
    \ifthenelse{\boolean{showannotations}}%
    {\rem{#1}~\add{#2}}%
    {#2}%
}



%%%%%%%%%%%%%%%%%%% Preprint info commands %%%%%%%%%%%%%%%%%%%
\newcommand{\placetextbox}[3]{% \placetextbox{<horizontal pos>}{<vertical pos>}{<stuff>}
  \setbox0=\hbox{#3}% Put <stuff> in a box
  \AddToShipoutPictureFG*{% Add <stuff> to current page foreground
    \put(\LenToUnit{#1\paperwidth},\LenToUnit{#2\paperheight}){\vtop{{\null}\makebox[0pt][c]{#3}}}%
  }%
}%

\newcommand{\preprintlabel}[2]{
    \placetextbox{0.5}{0.99}{\colorbox{gray!10}{\textcolor{WildStrawberry}{\textbf{{#2}}}}}%

    \placetextbox{0.5}{0.97}{\colorbox{gray!10}{\textcolor{WildStrawberry}{\textbf{Author pre-print. The final publication is available at: \url{https://doi.org/#1}.}}}}%
    
    \placetextbox{0.5}{0.05}{\colorbox{gray!10}{\textcolor{WildStrawberry}{\textbf{Author pre-print. The final publication is available at: \url{https://doi.org/#1}.}}}}%
}
%%%%%%%%%%%%%%%%%%% Show/hide annotations %%%%%%%%%%%%%%%%%%%
\newboolean{showannotations}
\setboolean{showannotations}{true} % TOGGLE TO SHOW ANNOTATIONS

%%%%%%%%%%%%%%%%%%% Show/hide names for double blind reviews %%%%%%%%%%%%%%%%%%%
\newboolean{shownames}
\setboolean{shownames}{true}


%%%%%%%%%%%%%%%%%%% Colors %%%%%%%%%%%%%%%%%%%
\newcommand{\newtextcolor}{blue}
\newcommand{\assignementcolor}{blue}

\title{latex-baseline}
\author{Istvan David}
\date{July 2025}

\begin{document}

\preprintlabel{xx.xxxx/xxxx}{\textbf{Here's an example of pre-print labeling. Use the \textbackslash\texttt{preprintlabel} command.}}


\maketitle

\section{Structure}
\begin{itemize}
    \item \verb|Preamble|
    \begin{itemize}
        \item \verb|packages.tex|: packages are imported here.
        \item \verb|commands.tex|: commands are defined here.
        \item \verb|settings.tex|: settings are defined here.
    \end{itemize}
\end{itemize}

\section{Settings}

The settings can be updated in the \verb|settings.tex| file.

\subsection{Margins}
Update \verb|\usepackage[margin=1in, top = 1in, bottom = 0.8in, headheight = 0.3in]{geometry}| as needed.

Update \verb|\setlength\parindent{0pt}| as needed.


\subsection{Anonymize paper for double blind reviews}
Replace \verb|\author{Steve}| with the following:
\begin{verbatim}
\author{
    \ifthenelse{\boolean{shownames}}%
    {Steve}%
    {Anonymous author(s)}%
}
\end{verbatim}

Use this block for other sensitive information. Change the \verb|\setboolean{shownames}{true}| setting to \verb|\setboolean{shownames}{false}|.

\subsection{Colors}

The commands in \secref{sec:commands} often use colors. These can be changed by modifying the following settings.
\begin{itemize}
    \item \verb|\newcommand{\newtextcolor}{blue}| -- sets the color of the added text annotations.
    \item \verb|\newcommand{\assignementcolor}{blue}| -- sets the color of the assignment annotations.
\end{itemize}





\section{Commands}\label{sec:commands}

\subsection{Frequently used annotations}

\begin{itemize}
    \item This is normal text.
    \item \TODO{This is a todo.}
    \item \TODO{} (Empty todo label.)
    \item \assignedto{Steve}
\end{itemize}

\hsep

\begin{itemize}
    \item Separating parts of texts might be useful occasionally. Use the \texttt{\textbackslash hsep} comment for the horizontal separator above.
\end{itemize}

\subsection{Review commands}

\begin{itemize}
    \item \rephrase{Should be rephrased}
    \item \id{This is my comment with my initials}
    \item \re{R1.01}{This marks a comment related to a reviewer's specific remark}
    \item \hl{This is a highlighting from the \texttt{soul} package for anything else.}
\end{itemize}

\subsection{Revision annotations}

\begin{itemize}
    \item \rem{Text removed.}
    \item \add{Text added.}
    \item \rep{Text replaced.}{New text.}
    \item \rem{Text with a reference{~\cite{bostrom2003are}} removed. -- Use braces around \texttt{$\sim$\textbackslash cite}.}
    \item \add{The \texttt{\textbackslash add\{\}} command does \textbf{NOT} support multiple paragraphs.}
    \item \addblockbegin{} Marking larger bodies of text as new is more efficient with placing it between a \texttt{\textbackslash addblockbegin\{\}} and \texttt{\textbackslash addblockend\{\}}. \addblockend{}
    
\end{itemize}

Some journals require submitting camera-ready revisions, but attaching the annotated version is advisable to enable traceability between the rebuttal and the revised manuscript, thus expiding things. Change the \verb|\setboolean{showannotations}{true}| setting to \verb|\setboolean{showannotations}{false}| to hide \verb|\add{}|, \verb|\rem{}| and \verb|\rep{}| annotations and produce the final version.


\subsection{Enumitem is loaded}

\begin{itemize}[noitemsep,topsep=0pt]
    \item Use \verb|\begin{itemize}[noitemsep,topsep=0pt]| for listst with a dense layout.
\end{itemize}

\subsection{Citing}

Natbib is loaded.
\begin{itemize}
    \item Use \verb|\citep{}| or \verb|\cite{}| for regular citations: \cite{bostrom2003are}.
    \item Use \verb|\citet{}| for author name citations: \citet{bostrom2003are}.
\end{itemize}

\bibliographystyle{IEEEtranN}
\bibliography{bib/references}

\end{document}
